\documentclass{beamer}
\usetheme{Madrid}
\setbeamercovered{invisible}
\setbeamertemplate{navigation symbols}{}
%
\usepackage{graphicx}
\usepackage{algorithm}
\usepackage[noend]{algpseudocode}
\usepackage{beamerthemesplit}
%
\title[Florida International University]{SSD-based Energy Efficient Cloud Storage}
\author{Salma Rodriguez}
\institute[University] {
Florida International University \\
\medskip
{\emph{srodr063@fiu.edu}}}
\date{\today}
%
\begin{document}
%
\begin{frame}
\titlepage
\end{frame}
%
\section{Motivation}
\begin{frame}
\frametitle{Motivation}
\begin{block}
{Solid State Technology}
High capacity EEPROM devices for caching disk data
reduces energy consumption on storage server by moving
data on disk to the more energy-efficient flash memory.
\end{block}
\begin{block}
{Distributed SSD Caching}
We want to reduce energy consumption on distributed systems by
exploring the properties of dynamic spin control of storage
server disks and replication.
\end{block}
\end{frame}
%
\section{Dynamic Disk Spin}
%\Let{$k$}{current time in seconds}
%\Let{$c$}{time since last cache miss}
\begin{frame}
% \begin{algorithm}
\frametitle{Disk Spin: Implementation}
\bf Algorithm 1 \rm Spinning the disk up or down dynamically
%\caption{Algorithm for dynamic spinning the disk up or down
%\label{alg:dyn-spinning}}
\begin{algorithmic}[1]
\Procedure{Spin Up or Down}{}
\While {true}
 \If{disk is spinning}
  \State $k\gets$ current time in seconds
  \State $c\gets$ time since last cache miss
  \If{$c$ + $20 \leq k$}
   \State spin down the disk and change state to spinning
  \EndIf
 \Else \Comment{disk is not spinning}
  \If{DM Cache is blocking on a cache miss}
   \State spin up the disk and change state to not spinning
   \State unblock DM Cache
  \EndIf
 \EndIf
\EndWhile
\EndProcedure
\end{algorithmic}
% \end{algorithm}
\end{frame}
%
\section{Replication}
\begin{frame}
\frametitle{Consistent Hashing}
Idea: replicate data evenly with as little disruption
as possible when nodes join and exit a network. \\
\begin{theorem}
For any set of N nodes and K keys, with high probability:
\newline\newline
1. Each node is responsible for at most (1 + $\epsilon$)K/N keys
\newline\newline
2. When an (N + 1)st node joins or leaves the network,
   responsibility for O(K/N) keys changes hands (to or from
   the joining or leaving node)
\end{theorem}
Here $\epsilon$ may vary but has an upper bound of \textit{O(log N)}.
\end{frame}
%
%section{Verbatim}
%begin{frame}[fragile]
%frametitle{Verbatim}
%begin{example}[Putting Verbatim]
%begin{verbatim}
%begin{frame}
%frametitle{Outline}
%begin{block}
%Why Beamer?}
%
%end{block}
%
%end{frame}\end{verbatim}
%end{example}
%end{frame}
%
%begin{frame}[fragile]
%
%xample of the \verb|\cite| command to give a reference:
%xample of citation using \cite{key1} follows on.
%end{frame}
%
\section{References}
\begin{frame}
\frametitle{References}
\footnotesize {
\begin{thebibliography}{99}
 \bibitem[Label1, 2010]{key1} Author's name (1987)
 \newblock Title of the paper.
 \newblock \emph{Journal Name} 55(4), 765 -- 799.
\end{thebibliography}
}
\end{frame}
%
\end{document}
